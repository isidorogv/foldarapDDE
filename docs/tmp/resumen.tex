\pagenumbering{roman} \setcounter{page}{1}
\chapter*{Resumen\markboth{Resumen}{Resumen}}

El trabajo que aqu� se va a presentar trata sobre la implementaci�n de un programa que permita a un robot seguir un objeto de forma aut�noma. El robot empleado como plataforma hardware de prueba y ensayo es \emph{Rovio}. Para ello se utilizar� la biblioteca de visi�n artificial \emph{OpenCV}, que sera la encargada de procesar la informaci�n proveniente de la c�mara WiFi de Rovio en un ordenador port�til. Como sistema operativo de base para el desarrollo de esta aplicaci�n se utilizara GNU/Linux, en concreto la distribuci�n \emph{Ubuntu Lucid}. La aplicaci�n ser� desarrollada enteramente en lenguaje C. La situaci�n de partida es el robot ya configurado seg�n las instrucciones que figuran en su manual.\\
\\
Para acometer tal tarea, se ha estructurado el trabajo de la siguiente manera: Tras realizar una peque�a introducci�n en el primer cap�tulo y declarar los objetivos del trabajo en el segundo, en el tercero se expone de forma breve una visi�n general de la plataforma hardware empleada en el trabajo, es decir, se describen las partes constituyentes de Rovio, as� como su funcionamiento.\\
\\
En el cap�tulo cuatro se contin�a la exposici�n del trabajo con una explicaci�n, tambi�n breve, de la biblioteca OpenCV, as� como del API de Rovio, para continuar con una descripci�n del software de partida para este trabajo. Este software fue desarrollado tambi�n en el Laboratorio de Rob�tica de la Universidad de Le�n por un par de antiguos alumnos y se ha tomado como base para aprovechar las funciones de gesti�n de conexi�n mediante sockets que incluye.\\
\\
Se finalizar� el trabajo con un cap�tulo de conclusiones y propuestas para trabajos posteriores que puedan o pretendan utilizar todo el material desarrollado en este trabajo como base de inicio. Un ap�ndice anexado al final del trabajo aclarar� los problemas o dudas que puedan surgir en relaci�n con el fichero de configuraci�n de la aplicaci�n de control de Rovio.

